% %-նշանից հետո գրվածը մեկնաբանություններ են
\documentclass{article}

% լուսանցքի սահմանում
\usepackage[left=1cm,right=1cm,top=1cm,bottom=1.2cm,bindingoffset=0cm]{geometry}

% հայերեն տեքստի հետ աշխատելու համար
\usepackage{fontspec}
\setmainfont{DejaVu Serif}

% լրացուցիչ աշխատանք մաթեմատիկայի հետ
\usepackage{amsfonts,amssymb,amsthm,mathtools} % AMS
\usepackage{amsmath}
\usepackage{upgreek}
\usepackage{icomma} % "խելացի" ստորակե, $0,2$ --- թիվ, $0, 2$ --- թվարկում
\usepackage{mathrsfs} % Սիրուն տառատեսակ

% հիպերհղում
\usepackage[hidelinks, colorlinks=true,linkcolor=blue]{hyperref}

% բանաձևերի (չ)համարակալում
\mathtoolsset{showonlyrefs=true} % Ցույց տալ բանաձևների միայն այն համարները, որոնց տեքստում կա \eqref{} (հղում)

% գույների և այլ գրաֆիկական ձևափոխությունների համար    
\usepackage{xcolor}    
\usepackage{blindtext}
\usepackage{tcolorbox}
\usepackage{graphicx}
\usepackage{tikz} 

\renewcommand{\labelenumii}{\arabic{enumi}.\arabic{enumii}}
\renewcommand{\labelenumiii}{\arabic{enumi}.\arabic{enumii}.\arabic{enumiii}}
\renewcommand{\labelenumiv}{\arabic{enumi}.\arabic{enumii}.\arabic{enumiii}.\arabic{enumiv}}

\begin{document}

% \begin{tcolorbox}[width=\textwidth, colback={white}, colframe=teal]    
% Շրջանակների մեջ կարելի է վերցնել tcolorbox կարգավորման միջոցով, իսկ xcolor փաթեթը թույլ է տալիս գույների լայն ընտրություն։
% \end{tcolorbox}

% \begin{tcolorbox}[width=\textwidth, colback={white}, colframe=orange]
%    graphicx և tikz փաթեթները հնարավորություն են տալիս ստեղծել շատ տարբեր հավես պատկերներ և գծագրեր, ստորև նայենք օրինակներ։
% \end{tcolorbox}
Դաս $2$\\
\vspace{1cm}
\\
$A$ բազմության ճշգրիտ վերին եզրը $\sup A$-ով ենք նշանակում և անվանում ենք «սուպրեմում», իսկ $\inf A$-ով՝ ճշգրիտ ստորին եզրը, որն անվանում ենք «ինֆիմում»։\\
\vspace{1cm}
\begin{enumerate}
    \item Նշված պնդումներից, որն է ճշմարիտ, պատասխանը հիմանվորել։
    \begin{enumerate}
        \item $\exists a, b \in \mathbb{Q}: \; a + b \in \mathbb{I} = \mathbb{R} \setminus \mathbb{Q} = \{x \in \mathbb{R}: \; x \notin \mathbb{Q}\}$,
        \item $\exists a, b \in \mathbb{I}: \; a + b \in \mathbb{Q}$,
        \item $\forall a, b \in \mathbb{I}: \; a + b \notin \mathbb{N}$,
        \item $\forall a, b \in \mathbb{I}: \; a + b \in \mathbb{N}$,
    \end{enumerate}
    \item Դիցուք՝ $f(x) = \sin x$։ Պատկերել հետևյալ ֆունկցիաների գրաֆիկները $x \geq 0$ բազմության վրա, և հիմնավորել արդյունքը․
    \begin{enumerate}
        \item $g(x) = \sup f^{-1}([0, x)) + \inf f^{-1}([0, x))$,
        \item $g(x) = \sup f^{-1}([0, x)) \cdot \inf f^{-1}([0, x))$,
        \item $g(x) = \sup f^{-1}([0, x)) \cdot \inf f^{-1}((-x, 0])$։
    \end{enumerate}
    \item Հիմանավորել հետևյալ պնդումները։
    \begin{enumerate}
        \item Կան ներդրված ինտերվալներ, որ չունեն ընդհանուր կետ,
        \item Ռացիոնալ թվերը հաշվելի են։
    \end{enumerate}
\end{enumerate}

\end{document}