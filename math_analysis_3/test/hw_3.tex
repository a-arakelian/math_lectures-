% %-նշանից հետո գրվածը մեկնաբանություններ են
\documentclass{article}

% լուսանցքի սահմանում
\usepackage[left=1cm,right=1cm,top=1cm,bottom=1.2cm,bindingoffset=0cm]{geometry}

% հայերեն տեքստի հետ աշխատելու համար
\usepackage{fontspec}
\setmainfont{DejaVu Serif}

% լրացուցիչ աշխատանք մաթեմատիկայի հետ
\usepackage{amsfonts,amssymb,amsthm,mathtools} % AMS
\usepackage{amsmath}
\usepackage{upgreek}
\usepackage{icomma} % "խելացի" ստորակե, $0,2$ --- թիվ, $0, 2$ --- թվարկում
\usepackage{mathrsfs} % Սիրուն տառատեսակ

% հիպերհղում
\usepackage[hidelinks, colorlinks=true,linkcolor=blue]{hyperref}

% բանաձևերի (չ)համարակալում
\mathtoolsset{showonlyrefs=true} % Ցույց տալ բանաձևների միայն այն համարները, որոնց տեքստում կա \eqref{} (հղում)

% գույների և այլ գրաֆիկական ձևափոխությունների համար    
\usepackage{xcolor}    
\usepackage{blindtext}
\usepackage{tcolorbox}
\usepackage{graphicx}
\usepackage{tikz} 

\renewcommand{\labelenumii}{\arabic{enumi}.\arabic{enumii}}
\renewcommand{\labelenumiii}{\arabic{enumi}.\arabic{enumii}.\arabic{enumiii}}
\renewcommand{\labelenumiv}{\arabic{enumi}.\arabic{enumii}.\arabic{enumiii}.\arabic{enumiv}}

\begin{document}

% \begin{tcolorbox}[width=\textwidth, colback={white}, colframe=teal]    
% Շրջանակների մեջ կարելի է վերցնել tcolorbox կարգավորման միջոցով, իսկ xcolor փաթեթը թույլ է տալիս գույների լայն ընտրություն։
% \end{tcolorbox}

% \begin{tcolorbox}[width=\textwidth, colback={white}, colframe=orange]
%    graphicx և tikz փաթեթները հնարավորություն են տալիս ստեղծել շատ տարբեր հավես պատկերներ և գծագրեր, ստորև նայենք օրինակներ։
% \end{tcolorbox}
Դաս $3$\\
\vspace{2cm}
\begin{enumerate}
    \item Ցույց տալ որ․
    \begin{enumerate}
        \item $\sum_{i = 0}^n \frac{n!}{i!(n-i)!} = 2^n$,
        \item $\sum_{i = 0}^{n} \frac{(2n+1)!}{i!(2n-i+1)!} = 2^{2n}$,
        \item $\binom{n-1}{k-1} + \binom{n-1}{k} = \binom{n}{k}$,
        \item $\binom{k}{k} + \binom{k+1}{k} + \cdots + \binom{n}{k} = \binom{n+1}{k+1}$:
    \end{enumerate}
    \item Հաշվել սահմաները (պատասխանը հիմանավորել)․
    \begin{enumerate}
        \item դիցուք՝ $\lim_{n\rightarrow \infty}a_n = \lim_{n\rightarrow \infty}c_n$ և $ a_n \leq b_n \leq c_n$, գտնել $\lim_{n\rightarrow \infty}b_n$
        \item դիցուք՝ $a > 0$, $a^{\frac{1}{n}}$,
        \item դիցուք՝ $\lim_{n\rightarrow \infty}a_n = \alpha$ և $\lim_{n\rightarrow \infty}b_n = \beta$, գտնել $c_n = a_nb_{n+1}$ սահմանը։
        \item $a_n = \sum_{i=1}^n \frac{9}{10^i}$,
        \item $a_n = \frac{n}{2^n}$,
        \item $a_{n+1} = \frac{1+a_n}{2}$,
        \item $a_{n+1} = \frac{31}{10} a_n (1- a_n)$:
    \end{enumerate}
    \item Հիմանավորել հետևյալ պնդումները։
    \begin{enumerate}
        \item Եթե սահմանափակ հաջորդականությունը ունի միայն մեկ մասնակի սահման, ապա այն զուգմետ է։
        \item Կա հաջորդականություն, որ ունի միայն մեկ մասնակի սահման և զուգամետ չէ։
    \end{enumerate}
\end{enumerate}

\end{document}